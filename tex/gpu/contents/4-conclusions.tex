%-----------------------------------------------------------------
%	CONCLUSIONS
%	!TEX root = ./../main.tex
%-----------------------------------------------------------------
\section{Conclusions}

During the resolution of this exercise, we have had the opportunity to manually create a parallel code, in contrast with the one we had optimised in the OpenMP assignment, that was itself an improvement from the very first code of the course. This time we have used a different methodology to paralellise the code, we have learned to take advantage of the GPU of our system, whose usage has increased dramatically over the past few years.

All of this has let us comprehend the importance of parallel programming when aiming for high performance computation, for we have experienced an incremental increase of the performance in the code we have been improving.

One of the most important things learn in this assignment has been the fact that one should not use OpenACC as a magic formula for improving the execution time of the code; one has to be aware of how the machine is working and optimise the code to really take advantage of the hardware capabilities of the system.
