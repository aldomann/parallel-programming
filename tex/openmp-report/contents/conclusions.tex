%-----------------------------------------------------------------
%	CONCLUSIONS
%	!TEX root = ./../main.tex
%-----------------------------------------------------------------
\section{Conclusions}

During the resolution of this exercise, we have had the opportunity to further improve the code of the first exercise of the course. This time, though, we have used a different methodology to paralellise the code, the widely employed \emph{OpenMP}. Moreover, we have got started in the analytical tool kits known as \emph{TAU} and we have learnt the basics to understand and interpret different visalisation techniques for performance measurement, profiling, and tracing on parallelised code.

All of this has let us understand the importance of parallel programming when aiming for high performance computation, for we have experienced the increase of performance in our code when applying such techniques.
